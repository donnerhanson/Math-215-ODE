%----------------------------------------------------------------------------------------
%	PACKAGES AND DOCUMENT CONFIGURATIONS
%----------------------------------------------------------------------------------------

\documentclass{article}
\usepackage{natbib} % Required to change bibliography style to APA
\usepackage{amsmath} % Required for some math elements 
\usepackage{amssymb} %Required for math elements
\usepackage{enumitem}
\usepackage{tabto}
\usepackage{tikz}

\usepackage{listings}
\usepackage{xcolor}

\usepackage{float}


\usepackage{xspace}

\usepackage{mathtools}

\usepackage[T1]{fontenc}
\usepackage{titling}

\setlength{\droptitle}{-10em}   % This is your set screw

\pagestyle{empty}
%----------------------------------------------------------------------------------------
%	DOCUMENT INFORMATION
%----------------------------------------------------------------------------------------

\title{Separable Differential Equations and Homogeneous First-Order DE’s} % Title

\author{Donner \textsc{Hanson}\\
\and
Jack \textsc{McGrath}\\
\and
Sarah \textsc{Townsend}
\and
Anna \textsc{Greene}
\and
Caitlin \textsc{Felts}
} % Author name

\date{\today} % Date for the report

\begin{document}
\maketitle % Insert the title, author and date

\begin{center}
\begin{tabular}{l r}
Instructor: & Dr. Katherine Evans\\% Instructor/supervisor
Class: & Linear Algebra and Differential Equations\\ % Class
Time: & Monday, Wednesday, Friday 12:00PM - 1:00PM\\
\end{tabular}
\end{center}

%----------------------------------------------------------------------------------------
%	SECTION 1
%----------------------------------------------------------------------------------------
\begin{flushleft}
\section*{1.4 Separable Differential Equations  } {
Definition 1.4.1:
A first-order differential equation is called separable if it can be written in the form
\[p(y)\cdot \frac{dy}{dx} = q(x) \] 


\subsection*{\textbf{1.4.2 Theorem:}} 
If p(y) and q(x) are continuous, then Equation (1.4.1) has the general solution

\[\int q(y) \cdot dy  = \int p(x) \cdot dx + C\]\\


where $C$ is an arbitrary constant.\\
%%%%%%%%%%%%%%%%
%% EXAMPLE 1 SDE
%%%%%%%%%%%%%%%%
\bigskip
\subsection*{\textbf{Example 1:}} 
\textbf{Find the general solution to $\frac{dy}{dx} = y^2 \cdot e^x dx$}\\

Start by separating the variables to separate sides of the equation\\
\begin{align*}
\frac{1}{y^2} dy & = e^x dx
\end{align*}
Integrate both sides
\begin{align*}
\int \frac{1}{y^2} dy & = \int e^x dx \\
\int y^{-2} dy & = \int e^x dx \\
-y^{-1} & = e^x + C \\
-\frac{1}{y} & = e^x +C \\
y & = - \frac{1}{e^x + C} \\
\end{align*}

\subsection*{\textbf{Example 2:}} 
Find the solution to \[\frac{dy}{dx}=\frac{4yx}{x^2+8}\] \\ 
Separate the variables to each side of the equation \\

\begin{align*}
\frac{1}{4y} dy & = \frac{x}{x^2 + 8} dx
\end{align*}
Integrate both sides

\begin{align*}
\int \frac{1}{4y} dy & = \int \frac{x}{x^2 + 8} dx
\end{align*}

Use U-substitution for the integration of the right hand side.
\begin{align*}
u & = x^2 + 8  && dx = \frac{du}{2x}
\end{align*}

Obtaining: \\
\begin{align*}
\frac{1}{4}\int\frac{1}{y}dy & = \int \frac{x}{u} \cdot \frac{du}{2x} \\
\frac{1}{4}\int\frac{1}{y}dy & = \int \frac{1}{u} \cdot \frac{du}{2} \\
\frac{1}{4}\int\frac{1}{y}dy & = \frac{1}{2} \int \frac{1}{u} du \\
\frac{1}{4}ln\mid y \mid & = \frac{1}{2} ln \mid u \mid + C
\end{align*}

Replace $u$ in terms of $x$ back into the equation to get in original terms of $x$ and $y$\\
\begin{align*}
\frac{1}{4}ln\mid y \mid & = \frac{1}{2} ln \mid (x^2 + 8) \mid + C \\
ln\mid y \mid & = \frac{4}{2} ln \mid (x^2 + 8) \mid + C \\
ln\mid y \mid & = \frac{4}{2} ln (x^2 + 8) + C \\
e^{ln(y)} & = e^{2ln(x^2+8) + C} \\
y & = e^{ln((x^2 + 8)^2)}\cdot e^C \\
y & = (x^2 + 8)^2 \cdot e^C \\
\text {\textbf{let: } } C_1 & = e^C \\
y & = C_1(x^2 + 8)^2 
\end{align*}

\subsection*{\textbf{Example 3:}}
At 2PM on a cool ($34^{\circ}$F)  afternoon in March, Sherlock Holmes measured the temperature of a dead body to be at $38^{\circ}$F. One hour later, the temperature was at $38^{\circ}$F. After a quick calculation using Newton’s law of cooling and taking the normal temperature of a living body to be $98^{\circ}$F, Holmes concluded that time of death was 10 AM. Was Holmes right?\\
\bigskip
To solve this problem we can use Newton's Law of Cooling.\\
\[ \frac{dT}{dt} = -k(T-T_m)\]

\[ T_m = 34 \]

Setup:
\begin{align*}
\frac{dT}{dt} & = -k(T-34) && \text{[Plug in known variables]}\\
\int \frac{dT}{(T-34)} & = \int -kdt && \text{[Integrate both sides]}
\end{align*}
\bigskip
Solve for T:\\
\begin{align*}
ln(T - 34) & =  -kt + C \\
e^{ln(T - 34)} & = e^{-kt + C} \\
T - 34 & = C_1e^{-kt} && \text{[let: $e^C = C_1$]}\\
T & = C_1e^{-kt} +34
\end{align*}
Solve for the given times to find k. \\
Note: t is found by taking time elapsed from t = 0 (10 AM).\\
At 2 PM the body's temperature was $38^{\circ}$F\\
\begin{align*}
T & = C_1e^{-kt} + 34 && \text{[Using the equation]} \\
38 & = C_1e^{-k(4)} + 34 && \text{[t = 4, hours between 10 AM and  2 PM]} \\
4 & = C_1e^{-4k}
\end{align*}
At 3 PM the body's temperature was $36^{\circ}$F\\
\begin{align*}
36 & = C_1e^{-k(5)} + 34 && \text{[t = 5, hours between 10 AM and  3 PM]} \\
2 & = C_1e^{-5k}
\end{align*}
Solve for k.\\
\begin{align*}
\frac{4}{2} & = \frac{C_1e^{-4k}}{C_1e^{-5k}}\\
2 & = e^k\\
ln(2) & = k
\end{align*}
Plug in k to solve for $C_1$ which is equivalent to $e^C$
\begin{align*}
4 & = C_1e^{-4ln(2)} \\
4 & = \frac{1}{16}C_1 \\
C_1 & = 64 
\end{align*}
Substitute for $t = 0$ in the initial value equation to check if Holmes was correct about the crime occurring at 10 AM.
\begin{align*}
T(0) & = 34 + 64 \cdot e^{(0)(ln2)}\\
T(0) & = 34 + 64\\
T(0) & = 98 
\end{align*}
\bigskip
Since the temperature at 10 AM was determined to be $98^{\circ}$ Holmes was correct!\\

}
%%%%%%%%%%%%%%%%
%%HOMOGENOUS F-O DE %
%%%%%%%%%%%%%%%%
\section*{1.8 Homogeneous first-order DE’s}{

\subsection*{\textbf{1.8.1 Definition:}}
A function is homogeneous of degree zero if\\

\[f(tx,ty) = f(x,y)\] \\

for all positive values of $t$ for which $(tx , ty)$ is in the domain of $f$.\\


\subsection*{\textbf{1.8.3 Theorem:}}
 A function $f(x,y)$ is homogeneous of degree zero if and only if it depends on the ratio of $\frac{y}{x}$ or $\frac{x}{y}$ only. \\

\subsection*{\textbf{1.8.4 Definition:}}
If $f(x,y)$ is homogeneous of degree zero, then the differential equation 
\[ \frac{dy}{dx} = f(x,y) \]\\
is called a homogeneous first-order differential equation. 

\subsection*{\textbf{1.8.5 Theorem:}}
The change of variables $ y = x \cdot V(x) $ reduces a homogeneous first-order differential equation 
$ \frac{dy}{dx} = f(x,y) $ to the separable equation 

\[\frac{1}{F(V) - V} \cdot dV = \frac{1}{x} \cdot dx \]\\

\textbf{Form of a First-Order Homogenous Differential Equation:}\\

\[ \frac{dy}{dx} = F (\frac{y}{x}) \]

\textbf{Steps in solving a First-Order Homogenous DE}\\

\begin{enumerate}
\item perform substitution: $ v = \frac{y}{x} \rightarrow y = x \cdot v$ and $\frac{y}{x} = x \cdot \frac{dv}{dx} + v $
\item Solve the differential equation using separation of variables
\item Solve the original differential equation in terms of $x$ and $y$
\end{enumerate}
\bigskip
\subsection*{\textbf{Example 1:}}
Determine if the following equation is homogeneous. If it is homogeneous, then solve.
\begin{align*}
\frac{dy}{dx} & = \frac{3y^2 +xy}{x^2}\\
\end{align*}
Multiplying both the numerator and the denominator by $\frac{1}{x^2}$ results in:
\begin{align*}
\frac{dy}{dx} & = 3(\frac{y}{x})^2+\frac{y}{x}\\
\end{align*}
Which is in the form of:
\begin{align*}
\frac{dy}{dx} & = F (\frac{y}{x})
\end{align*}

Using our substitutions:
\begin{align*}
v & = \frac{y}{x}\\
\frac{dy}{dx} & = x \frac{dv}{dx} + v
 \end{align*}
We obtain:
\begin{align*}
x \cdot \frac{dv}{dx} + v & = 3v^2 + v\\
x \cdot  \frac{dv}{dx}  & = 3v^2 
\end{align*}

Rearrange to solve using separable DE method
\begin{align*}
\frac{1}{3v^2} \cdot dv & = \frac{1}{x} \cdot dx \\
\frac{1}{3} v^{-2} \cdot dv & = \frac{1}{x} \cdot dx \\
\int \frac{1}{3} v^{-2} \cdot dv & = \int \frac{1}{x} \cdot dx  \\
-\frac{1}{3}v^{-1} & = ln\mid x \mid + C
\end{align*}

Now acquire the equation in terms of x and y with our substitution ratios

\begin{align*}
-\frac{1}{3}v^{-1} & = ln\mid x \mid + C \\
-\frac{1}{3}(\frac{y}{x})^{-1} & = ln\mid x \mid + C \\
-\frac{1}{3}(\frac{x}{y}) & = ln\mid x \mid + C \\
-\frac{1}{3}(\frac{x}{y}) & = ln\mid x \mid + C \\
\frac{x}{y} & = -3 (ln\mid x \mid + C) \\
\frac{x}{y} & = \frac{ -3 (ln\mid x \mid + C) }{1} \\
\frac{y}{x} & = \frac{1}{ -3 (ln\mid x \mid + C) } \\
y & =  \frac{x}{ -3 (ln\mid x \mid + C) } 
\end{align*}


\subsection*{\textbf{Example 2:}}
Determine if the following equation is homogeneous. If it is homogeneous, then solve.
\begin{align*}
\frac{dy}{dx} & = \frac{y}{x +\sqrt{xy}}
\end{align*}
Multiplying both the numerator and the denominator by $\frac{1}{x}$ results in:\\
\begin{align*}
\frac{dy}{dx} & = \frac{(\frac{y}{x})}{\frac{x}{x} +\sqrt{\frac{xy}{x^2}}}
\end{align*}
To get this in the form of:
\begin{align*}
\frac{dy}{dx} & = F (\frac{y}{x})
\end{align*}
We need to simplify, which will give us:\\
\begin{align*}
\frac{dy}{dx} & = \frac{(\frac{y}{x})}{1 +\sqrt{(\frac{y}{x})}}
\end{align*}

Once in the correct form, using our substitutions:\\
\begin{align*}
v & = \frac{y}{x}\\
\frac{dy}{dx} & = x \frac{dv}{dx} + v
 \end{align*}
We obtain:\\
\begin{align*}
x \cdot \frac{dv}{dx} + v & = \frac{v}{1 + v^\frac{1}{2}}\\
x \cdot  \frac{dv}{dx}  & = \frac{v}{1 + v^\frac{1}{2}} - \frac{v}{1}\\
x \cdot  \frac{dv}{dx}  & = \frac{-v^\frac{3}{2}}{1 + v^\frac{1}{2}}
\end{align*}

Rearrange to solve using separable DE method\\
\begin{align*}
\frac{1 + v^\frac{1}{2}}{-v^\frac{3}{2}} \cdot dv & = \frac{1}{x} \cdot dx \\
(-v^\frac{-3}{2} - \frac{1}{v}) \cdot dv & = \frac{1}{x} \cdot dx \\
\int (-v^\frac{-3}{2} - \frac{1}{v}) \cdot dv & = \int \frac{1}{x} \cdot dx  \\
2v^\frac{-1}{2} - ln \mid v \mid & = ln \mid x \mid + C 
\end{align*}

Now acquire the equation in terms of x and y with our substitution ratios

\begin{align*}
2v^\frac{-1}{2} - ln \mid v \mid & = ln \mid x \mid + C \\
2(\frac{x}{y})^\frac{1}{2} - ln \mid \frac{y}{x} \mid & = ln \mid x \mid + C \\
2(\frac{x}{y})^\frac{1}{2} - (ln \mid y \mid - ln \mid x \mid) & = ln \mid x \mid + C \\
2\sqrt{\frac{x}{y}} - ln \mid y \mid & = C 
\end{align*}
\bigskip
\subsection*{\textbf{Example 3:}}
\begin{align*}
\frac{dy}{dx} & = \frac{x-y}{x+y}
\end{align*}
Get in the form of:\\
\begin{align*}
\frac{dy}{dx} & = F (\frac{y}{x})
\end{align*}
Divide the numerator and denominator by $x$:\\
\begin{align*}
\frac{dy}{dx} & = \frac{\frac{x}{x}-\frac{y}{x}}{\frac{x}{x}+\frac{y}{x}}
\end{align*}
Simplify the equation to the correct form:
\begin{align*}
\frac{dy}{dx} & = \frac{1-\frac{y}{x}}{1+\frac{y}{x}}
\end{align*}

Using our substitutions:
\begin{align*}
y & = v \cdot  x\\
\frac{1-v}{1+v} & = v + x(\frac{dv}{dx})
 \end{align*}
 Simplifying further to obtain:
\begin{align*}
x \frac{dv}{dx} & = \frac{1-2v-v^2}{1+v}
 \end{align*} 
 
 Separate and integrate:
\begin{align*}
\int \frac{1+v}{1-2v-v^2}dv & = \int \frac{1}{x}dx\\
-\frac{1}{2} ln(1-2v-v^2) & = ln(x) + C
 \end{align*} 
 
 Substitute ln(k) for C:
 \begin{align*}
-\frac{1}{2} ln(1-2v-v^2) & = ln(x) + ln(k) \\
 1-2v-v^2 & = \frac{1}{x^2k^2 }
 \end{align*}
 
Since we have simplified the equation $1-2v-v^2 =\frac{1}{x^2k^2 } $  replace $v = \frac{y}{x}$
 \begin{align*}
 1-2(\frac{y}{x})-(\frac{y}{x})^2 & = \frac{1}{x^2k^2 }\\
 x^2-2xy-y^2 & = \frac{1}{k^2 }\\
\end{align*}

Solve for $y$:

 \begin{align*}
x^2-2xy-y^2 & = \frac{1}{k^2 }\\
y^2+2xy-x^2 & = -\frac{1}{k^2 }\\
y^2+2xy-x^2 & = -\frac{1}{k^2 }
\end{align*}

Add $2x^2$ to both sides and consider $-\frac{1}{k^2}$ to be a constant $C_1$ so we can simplify:
\begin{align*}
y^2+2xy+x^2 & = 2x^2+C_1\\
(y+x)^2 & = 2x^2+C_1\\
y & = \pm \sqrt{2x^2+C_1} - x
\end{align*}


}


\end{flushleft}
%\newpage
%----------------------------------------------------------------------------------------


\end{document}\\