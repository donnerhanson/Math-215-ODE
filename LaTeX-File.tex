%----------------------------------------------------------------------------------------
%	PACKAGES AND DOCUMENT CONFIGURATIONS
%----------------------------------------------------------------------------------------

\documentclass{article}
\usepackage{natbib} % Required to change bibliography style to APA
\usepackage{amsmath} % Required for some math elements 
\usepackage{amssymb} %Required for math elements
\usepackage{enumitem}
\usepackage{tabto}
\usepackage{tikz}

\usepackage{listings}
\usepackage{xcolor}

\usepackage{float}


\usepackage{xspace}

\usepackage{mathtools}

\usepackage[T1]{fontenc}
\usepackage{titling}

\setlength{\droptitle}{-10em}   % This is your set screw

\pagestyle{empty}
%----------------------------------------------------------------------------------------
%	DOCUMENT INFORMATION
%----------------------------------------------------------------------------------------

\title{Separable Differential Equations and Homogeneous First-Order DE’s} % Title

\author{Donner \textsc{Hanson}\\
\and
Jack \textsc{McGrath}\\
\and
Sarah \textsc{Townsend}
\and
Anna \textsc{Greene}
\and
Caitlin \textsc{Felts}
} % Author name

\date{\today} % Date for the report

\begin{document}
\maketitle % Insert the title, author and date

\begin{center}
\begin{tabular}{l r}
Instructor: & Dr. Katherine Evans\\% Instructor/supervisor
Class: & Linear Algebra and Differential Equations\\ % Class
Time: & Monday, Wednesday, Friday 12:00PM - 1:00PM\\
\end{tabular}
\end{center}

%----------------------------------------------------------------------------------------
%	SECTION 1
%----------------------------------------------------------------------------------------
\begin{flushleft}
\section*{1.4 Separable Differential Equations  }{
Definition 1.4.1:
A first-order differential equation is called separable if it can be written in the form
\[p(y)\cdot \frac{dy}{dx} = q(x) \] 


\subsection*{\textbf{1.4.2 Theorem:}} 
If p(y) and q(x) are continuous, then Equation (1.4.1) has the general solution

\[\int q(y) \cdot dy  = \int p(x) \cdot dx + c\]\\


where $c$ is an arbitrary constant.

\textbf{[2-3 examples]}

}
\section*{1.8 Homogeneous first-order DE’s}{

\subsection*{\textbf{1.8.1 Definition:}}
A function is homogeneous of degree zero if\\

\[f(tx,ty) = f(x,y)\] \\

for all positive values of $t$ for which $(tx , ty)$ is in the domain of $f$.\\


\subsection*{\textbf{1.8.3 Theorem:}}
 A function $f(x,y)$ is homogeneous of degree zero if and only if it depends on $y$ or $x$ only. \\

\subsection*{\textbf{1.8.4 Definition:}}
If $f(x,y)$ is homogeneous of degree zero, then the differential equation 
\[ \frac{dy}{dx} = f(x,y) \]\\
is called a homogeneous first-order differential equation. 

\subsection*{\textbf{1.8.5 Theorem:}}
The change of variables $ y = x \cdot V(x) $ reduces a homogeneous first-order differential equation 
$ \frac{dy}{dx} = f(x,y) $ to the separable equation 

\[\frac{1}{F(V) - V} \cdot dV = \frac{1}{x} \cdot dx \]\\

\textbf{[2-3 examples]}
}
\end{flushleft}
%\newpage
%----------------------------------------------------------------------------------------


\end{document}\\